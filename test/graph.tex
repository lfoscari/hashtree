\documentclass[a4paper]{article}

\usepackage{pgfplots}
\pgfplotsset{compat=1.17}

% \usepackage[utf8]{inputenc}
% \usepackage{amsmath, amssymb}
% \usepackage[boxed]{algorithm2e}

% \setlength{\parindent}{0pt}
% \setlength{\parskip}{10pt}

\begin{document}

	I parametri in uso sono
	\begin{itemize}
		\item $n$ il numero di elementi inseriti.
		\item $m$ la dimensione massima delle foglie.
		\item $b$ la taglia delle tabelle di hash nei nodi.
	\end{itemize}
	Le metriche analizzate sono
	\begin{itemize}
		\item \textit{Usage} o utilizzo è la percentuale di riempimento delle foglie rispetto ad $m$.
		\item \textit{Depth} è la profondità dell'albero.
	\end{itemize}
	E sono analizzate dopo l'inserimento di $n$ elementi. È possibile fare le seguenti considerazioni su $m$ e $b$
	\begin{itemize}
		\item Riguardo a $b$ possiamo dire che
		\begin{itemize}
			\item Se $b = 1$ l'utilizzo è sempre al 100\%, perchè c'è una scissione ad ogni inserimento dopo l'$m$-esimo, mentre la profondità è $n - m + 1$.
			\item Se $b = n$ l'utilizzo ha un valore atteso pari a $n/m$ e la profondità $1$, ipotizzando che l'hashing utilizzato sia universale. 
		\end{itemize}
		\item Riguardo ad $m$ possiamo dire che
		\begin{itemize}
			\item Al crescere di $m$ da 1 ad $n$ la profondità e l'utilizzo tendono a scendere. 
			\item Dal momento in cui $m = n$ non avvengono scissioni, perciò la profondità sarà sempre 1 e l'utilizzo al 100\%.
		\end{itemize}
	\end{itemize}

	\pagebreak

	\section*{\texttt{MAGIC Gamma Telescope Dataset}}

	È stato utilizzato un subset di $n = 50$ righe del \texttt{MAGIC Gamma Telescope Dataset}. Per valori di $m$ e $b$ tra 1 e $n$ sono stati calcolati utilizzo e profondità medi su 10 permutazioni. Questi sono i risultati.

	\begin{center}
		\begin{tikzpicture}
			\begin{axis}[
					xlabel={$m$}, xmin=1, xmax=50,
					ylabel={$b$}, ymin=1, ymax=50,
					zlabel={Usage}, zmin=0.0, zmax=1.0, zmajorgrids=true,
					width=\textwidth * 0.9
				]
				\addplot3 [scatter, only marks, mark=+]
					table [x=m, y=b, z=avg_usage, col sep=comma] {../resources/magic/magic04.out};
			\end{axis}

			\begin{axis}[
					xlabel={$m$}, xmin=1, xmax=50,
					ylabel={$b$}, ymin=1, ymax=50,
					zlabel={Depth}, zmin=1, zmax=50, zmajorgrids=true,
					width=\textwidth * 0.9, yshift=9cm
				]
				\addplot3 [scatter, only marks, mark=+]
					table [x=m, y=b, z=avg_depth, col sep=comma] {../resources/magic/magic04.out};
			\end{axis}
		\end{tikzpicture}

		\begin{tikzpicture}
			\begin{axis}[
				xlabel={$m$}, xmin=1, xmax=50,
				ylabel={$b$}, ymin=1, ymax=50,
				zlabel={Number of accesses}, zmin=0.0, zmajorgrids=true,
				width=\textwidth * 0.9, yshift=18cm
			]
			\addplot3 [scatter, only marks, mark=+]
				table [x=m, y=b, z=avg_access, col sep=comma] {../resources/magic/magic04.out};
			\end{axis}
		\end{tikzpicture}
	\end{center}

	\pagebreak

	\section*{\texttt{Cloud DataSet}}

	È stato utilizzato un subset di $n = 50$ righe del \texttt{Cloud DataSet}. Per valori di $m$ e $b$ tra 1 e $n$ sono stati calcolati utilizzo e profondità medi su 10 permutazioni. Questi sono i risultati.

	\begin{center}
		\begin{tikzpicture}
			\begin{axis}[
					xlabel={$m$}, xmin=1, xmax=50,
					ylabel={$b$}, ymin=1, ymax=50,
					zlabel={Usage}, zmin=0.0, zmax=1.0, zmajorgrids=true,
					width=\textwidth * 0.9
				]
				\addplot3 [scatter, only marks, mark=+]
					table [x=m, y=b, z=avg_usage, col sep=comma] {../resources/cloud/cloud.out};
			\end{axis}

			\begin{axis}[
					xlabel={$m$}, xmin=1, xmax=50,
					ylabel={$b$}, ymin=1, ymax=50,
					zlabel={Depth}, zmin=1, zmax=50, zmajorgrids=true,
					width=\textwidth * 0.9, yshift=9cm
				]
				\addplot3 [scatter, only marks, mark=+]
					table [x=m, y=b, z=avg_depth, col sep=comma] {../resources/cloud/cloud.out};
			\end{axis}
		\end{tikzpicture}
	\end{center}


\end{document}